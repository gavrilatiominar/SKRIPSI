\documentclass[a4paper,twoside]{article}
\usepackage[T1]{fontenc}
\usepackage[bahasa]{babel}
\usepackage{graphicx}
\usepackage{graphics}
\usepackage{float}
\usepackage[cm]{fullpage}
\pagestyle{myheadings}
\usepackage{etoolbox}
\usepackage{setspace} 
\usepackage{lipsum} 
\setlength{\headsep}{30pt}
\usepackage[inner=2cm,outer=2.5cm,top=2.5cm,bottom=2cm]{geometry} %margin
% \pagestyle{empty}

\makeatletter
\renewcommand{\@maketitle} {\begin{center} {\LARGE \textbf{ \textsc{\@title}} \par} \bigskip {\large \textbf{\textsc{\@author}} }\end{center} }
\renewcommand{\thispagestyle}[1]{}
\markright{\textbf{\textsc{AIF401/AIF402 \textemdash Rencana Kerja Skripsi \textemdash Sem. Ganjil 2017/2018}}}

\onehalfspacing
 
\begin{document}

\title{\@judultopik}
\author{\nama \textendash \@npm} 

%tulis nama dan NPM anda di sini:
\newcommand{\nama}{Gavrila Tiominar Sianturi}
\newcommand{\@npm}{2013730025}
\newcommand{\@judultopik}{Steganografi dengan Teknik Indikasi Piksel} % Judul/topik anda
\newcommand{\jumpemb}{1} % Jumlah pembimbing, 1 atau 2
\newcommand{\tanggal}{13/09/2017}

% Dokumen hasil template ini harus dicetak bolak-balik !!!!

\maketitle

\pagenumbering{arabic}

\section{Deskripsi}
Steganografi merupakan salah satu teknik yang digunakan untuk menjaga keamanan suatu informasi dengan cara menyembunyikannya. Informasi tersebut dapat disembunyikan dalam berbagai bentuk media, seperti audio, video, gambar, dan lain sebagainya. Media yang digunakan untuk menyembunyikan informasi disebut \textit{cover media}. Tujuan dari steganografi adalah tidak terlihatnya perbedaan pada \textit{cover media} yang sudah dimodifikasi, sehingga tidak menimbulkan kecurigaan pada pihak ketiga. Pada skripsi ini, akan dibahas metode steganografi untuk menyembunyikan informasi pada \textit{cover media} berupa gambar. Gambar yang akan digunakan merupakan gambar berwarna yang terdiri dari 3 \textit{channel} warna, yaitu merah, hijau, dan biru.

Banyak metode dapat digunakan untuk menyembunyikan informasi dalam gambar. Salah satu metode yang sudah banyak digunakan adalah \textit{Least Significant Bits} (LSB) dan \textit{Stego Color Cycle} (SCC). Pada LSB, informasi disembunyikan pada bit terakhir dalam setiap piksel dari \textit{cover media}, baik pada suatu \textit{channel} warna atau pada seluruh \textit{channel} warna. Teknik LSB sangat sederhana, sehingga informasi yang disembunyikan lebih mudah terdeteksi. Teknik SCC merupakan pengembangan dari teknik LSB. Pada SCC, informasi disembunyikan pada \textit{channel} warna tertentu secara bergantian dengan pola yang sudah ditentukan. Contohnya, bit pertama informasi akan disembunyikan pada bit terakhir \textit{channel} merah pada piksel pertama, lalu bit kedua akan disembunyikan pada bit terakhir \textit{channel} hijau pada piksel kedua, bit ketiga akan disembunyikan pada bit terakhir \textit{channel} biru pada piksel ketiga, dan seterusnya. Deteksi dari pihak ketiga akan lebih sulit dari LSB. Namun apabila pola pergantian \textit{channel} warna yang digunakan sudah dapat diketahui, maka pihak ketiga akan dengan mudah mengetahui informasi yang disembunyikan tersebut.

Teknik lainnya yang merupakan pengembangan dari teknik LSB adalah teknik indikasi piksel. Pada teknik indikasi piksel, steganografi dilakukan dengan cara menggunakan paling sedikit dua bit dari salah satu \textit{channel} warna pada piksel \textit{cover media} sebagai indikator untuk menentukan apakah ada informasi yang disembunyikan pada dua \textit{channel} warna lainnya. Pemilihan indikator dilakukan secara acak, sehingga pihak ketiga sulit untuk mendeteksi informasi yang disembunyikan. Dalam pengembangannya, terdapat algoritma \textit{Triple-A} di mana informasi yang akan disembunyikan dienkripsi terlebih dahulu dan hasil enkripsinya dimasukkan ke dalam piksel-piksel gambar. Pemilihan \textit{channel} warna yang akan digunakan dan jumlah bit yang akan dimasukkan pada \textit{channel} warna yang terpilih pun dilakukan secara acak. Teknik ini meningkatkan kesulitan bagi pihak ketiga untuk mengetahui informasi yang disembunyikan.

Perangkat lunak yang dirancang pada skripsi ini akan mengaplikasikan steganografi dengan teknik indikasi piksel dan dengan algoritma \textit{Triple-A}. Perangkat lunak akan menerima \textit{input} berupa gambar dan teks yang akan disembunyikan, serta mengembalikan \textit{output} berupa gambar yang sudah dimodifikasi dengan metode indikasi piksel dan juga \textit{Triple-A}. Perbedaan hasil dari kedua metode tersebut akan terlihat dari \textit{output} yang dihasilkan.

\section{Rumusan Masalah}
Rumusan masalah yang akan dibahas pada skripsi ini antara lain adalah:
\begin{enumerate}
	\item Bagaimana cara kerja steganografi dengan teknik indikasi piksel dan algoritma \textit{Triple-A} pada gambar?
	\item Bagaimana cara mengaplikasikan steganografi dengan teknik indikasi piksel dan algoritma \textit{Triple-A} pada gambar tanpa menimbulkan perbedaan antara sebelum dan sesudah gambar dimodifikasi?
\end{enumerate}

\section{Tujuan}
Tujuan yang ingin dicapai dalam skripsi ini, berdasarkan rumusan masalah yang sudah ditentukan, adalah:
\begin{enumerate}
	\item Mempelajari cara kerja steganografi dengan menggunakan teknik indikasi piksel dan algoritma \textit{Triple-A} pada gambar.
	\item Membangun perangkat lunak yang mengaplikasikan steganografi dengan teknik indikasi piksel dan algoritma \textit{Triple-A} tanpa menimbulkan perbedaan antara sebelum dan sesudah gambar dimodifikasi.
\end{enumerate}

\section{Deskripsi Perangkat Lunak}
Perangkat lunak akhir yang akan dibangun akan memiliki fitur minimal sebagai berikut:
\begin{itemize}
	\item Pengguna akan dapat memasukkan informasi yang disembunyikan (berupa teks) dan media untuk menyembunyikannya (berupa gambar).
	\item Perangkat lunak akan memproses gambar dan teks tersebut dengan metode indikasi piksel dan \textit{Triple-A}.
	\item Perangkat lunak akan mengembalikan hasil berupa gambar hasil metode indikasi piksel dan hasil algoritma \textit{Triple-A}.
\end{itemize}

\section{Detail Pengerjaan Skripsi}
Bagian-bagian pekerjaan skripsi ini adalah sebagai berikut :
	\begin{enumerate}
		\item Melakukan studi literatur mengenai dasar-dasar steganografi, metode steganografi dengan teknik indikasi piksel, dan metode steganografi dengan algoritma \textit{Triple-A}.
		\item Mengimplementasikan teknik indikasi piksel dan algoritma \textit{Triple-A} secara manual.
		\item Melakukan analisis kebutuhan.
		\item Melakukan perancangan perangkat lunak.
		\item Mengimplementasikan teknik indikasi piksel dan algoritma \textit{Triple-A} pada perangkat lunak.
		\item Melakukan pengujian teknik indikasi piksel dan algoritma \textit{Triple-A}.
		\item Melakukan analisis terhadap hasil pengujian.
		\item Menulis dokumen skripsi
	\end{enumerate}

\section{Rencana Kerja}
Berikut merupakan persentase dari pengerjaan skripsi yang akan terbagi pada AIF401 Skripsi 1 dan AIF402 Skripsi 2.

\begin{center}
  \begin{tabular}{ | c | c | c | c | l |}
    \hline
    1*  & 2*(\%) & 3*(\%) & 4*(\%) &5*\\ \hline \hline
    1   & 15 & 15 &  &  \\ \hline
    2   & 5 & 5 &  &  \\ \hline
    3   & 10 & 10 &  &  \\ \hline
    4   & 10 &  & 10 &  \\ \hline
    5   & 25 &  & 25 &  \\ \hline
    6   & 10 &  & 10 &  \\ \hline
    7   & 5 &  & 5 &  \\ \hline    
    8	& 20 & 5 & 15 & Pendahuluan, dasar teori, dan analisis dikerjakan pada skripsi 1. \\ \hline
    Total  & 100  & 35  & 65 &  \\ \hline
                          \end{tabular}
\end{center}

Keterangan (*)\\
1 : Bagian pengerjaan Skripsi (nomor disesuaikan dengan detail pengerjaan di bagian 5)\\
2 : Persentase total \\
3 : Persentase yang akan diselesaikan di Skripsi 1 \\
4 : Persentase yang akan diselesaikan di Skripsi 2 \\
5 : Penjelasan singkat apa yang dilakukan di S1 (Skripsi 1) atau S2 (Skripsi 2)

\vspace{1cm}
\centering Bandung, \tanggal\\
\vspace{2cm} \nama \\ 
\vspace{1cm}

Menyetujui, \\
\ifdefstring{\jumpemb}{2}{
\vspace{1.5cm}
\begin{centering} Menyetujui,\\ \end{centering} \vspace{0.75cm}
\begin{minipage}[b]{0.45\linewidth}
% \centering Bandung, \makebox[0.5cm]{\hrulefill}/\makebox[0.5cm]{\hrulefill}/2013 \\
\vspace{2cm} Nama: \makebox[3cm]{\hrulefill}\\ Pembimbing Utama
\end{minipage} \hspace{0.5cm}
\begin{minipage}[b]{0.45\linewidth}
% \centering Bandung, \makebox[0.5cm]{\hrulefill}/\makebox[0.5cm]{\hrulefill}/2013\\
\vspace{2cm} Nama: \makebox[3cm]{\hrulefill}\\ Pembimbing Pendamping
\end{minipage}
\vspace{0.5cm}
}{
% \centering Bandung, \makebox[0.5cm]{\hrulefill}/\makebox[0.5cm]{\hrulefill}/2013\\
\vspace{2cm} Nama: \makebox[3cm]{\hrulefill}\\ Pembimbing Tunggal
}
\end{document}

